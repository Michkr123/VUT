% Tento soubor nahraďte vlastním souborem s obsahem práce.
%=========================================================================
% Autoři: Michal Bidlo, Bohuslav Křena, Jaroslav Dytrych, Petr Veigend a Adam Herout 2019

% Pro kompilaci po částech (viz projekt.tex), nutno odkomentovat a upravit
%\documentclass[../projekt.tex]{subfiles}
%\begin{document}

\chapter{Úvod}
Podpis je jednou z~nejstarších metod používanou pro ověření totožnosti, určení autorství či udělení právního souhlasu.
V~této práci půjde především o~ověření totožnosti, které spadá pod obor zvaný biometrie.
Biometrie se zabývá analýzou biologických a behaviorálních charakteristik používaných mimo jiné k~autentizaci. 
Zkoumáním pravosti podpisu a jiných textů se zabývá písmoznalectví neboli grafognózie.
U~podpisu lze analyzovat statické a dynamické parametry písma. 

%Čím se tedy zabývá tato práce?
Podpis je v~praxi stále nejpoužívanější způsob autentizace. 
Podaří-li se důvěryhodně podpis napodobit, bude možnost obejít bezpečnostní zabezpečení. 
To představuje velké bezpečnostní riziko a nutnost upravení dosavadních bezpečnostních metod pro autentizaci. 
Bavíme se zde spíše o~digitálním podpisu. 
Na rozdíl od podpisu na papíře digitální podpis obsahuje i dynamické charakteristiky.
Bude tedy o něco složitější jej zfalšovat.

%TOLE JE SPIS ABSTRAKT
%Bude třeba sestavit snímací zařízení pro získaní potřebných parametrů podpisu.
%Ze statických parametrů budeme snímat tvar i tloušťku čar a celkový vzhled podpisu. 
%Z dynamických parametrů budeme snímat tlak, sklon a průběh tahů. 
%Získané data je potřeba vhodným způsobem zpracovat a provést potřebné výpočty. 
%Upravená data je poté potřeba převést do instrukcí pro mechanickou paži, která je použita k replikaci podpisu. 
%Výsledný podpis bude porovnáván s vlastnoručním podpisem. Cílem je co nejvíce se přiblížit vlastnoručnímu podpisu.

\section{Pojmy}
\label{sec:pojmy}
\begin{itemize}
  \item{\textbf{Biologické autentizační metody} - zkoumají biologické charakteristiky člověka, s~jimiž se narodí. Tyto charakteristiky má každý člověk unikátní a jsou neměnitelné. Patří mezi ně například otisky prstů, skeny sítnice či duhovky.}  
  \item{\textbf{Behaviorální autentizační metody} - porovnávají behaviorální charakteristiky související s~chováním člověka. Lze například porovnávat, jakým způsobem jedinec mluví nebo jak pohybuje očima.}
  \item{\textbf{Statické parametry písma} - jde o~parametry, které nezahrnují informace o~procesu psaní podpisu. Patří mezi ně například tvar a vzhled písma, umístění podpisu na stránce nebo také tloušťka čar.}
  \item{\textbf{Dynamické parametry písma} - tyto parametry se vytahují k~samotnému průběhu psaní podpisu. Například rychlost psaní, akcelerace, tlak nebo průběh tahů.}
\end{itemize}
%\section{Představení tématu}
%\section{Cíle práce}

%\section{Struktura práce}

\chapter{Teoretické základy}
Pro plné pochopení problematiky bylo potřeba nastudovat podklady z~několika odvětví. 
V~této části budou shrnuty všechny potřebné informace. 
Pro snazší orientaci jsou níže rozděleny do bloků.

\section{Definice podpisu a jeho charakteristiky}
Definice podpisu je poněkud problematická. 
Podpis jako takový není v~zákoně nijak přímo definován. 
Obecně je ale bráno za podpis vlastnoruční uvedení jména a příjmení, nebo jiného jedinečného a nezaměnitelného označení. %https://www.fulsoft.cz/33/komentar-zakona-89-2012-sb-obcansky-zakonik-561-uniqueidmRRWSbk196FNf8-jVUh4EtuvCojfP1Dmx8t50X_QR_yU_dtwjPItMA/

Charakteristiky podpisu je nepřeberné množství informací, podle kterých lze jednotlivé podpisy od sebe odlišit. 
Charakteristiky se dělí na statické a dynamické~\ref{sec:pojmy}. 
Jsou ovlivňovány jak fyzickým, tak i psychologickým stavem člověka v době podpisu.
Mezi charakteristiky patří například celkový vzhled podpisu, tloušťka čar, rychlost psaní, tlak, průběh a pořadí tahů a mnoho dalších, které ale pro nás v rámci této práce nebudou důležité. %TODO vymyslel jsem si lol

\section{Biometrická autentizace}
Biometrická autentizace probíhá na základě biologických a behaviorálních charakteristik člověka\ref{sec:pojmy}. 
Obě tyto skupiny znaků jsou pro každého člověka unikátní a nezaměnitelné. To je důvodem, proč je lze využít k autentizaci.
Existuje spousta biometrických metod pro autentizaci, kupříkladu otisk prstu, rozpoznání obličeje, sken duhovky či sítnice, nebo dokonce pomocí používání myši. 
Každá tato metoda má své výhody a nevýhody, zejména se hodnotí přesnost, cena, komfort používání, stálost a velikost vzorku.
Tato práce se zaměřuje na autentizaci podpisem.

\section{Parametry podpisu řešené v rámci této práce}
V rámci této práce budou snímány následující parametry:
\begin{itemize}
  \item{\textbf{Celkový vzhled podpisu}}
  \item{\textbf{Pohyb pera} při psaní pomocí akcelerometru zabudovaném v MPU-6050 ze kterého bude vypočítán celkový průběh podpisu, především pak \textbf{pozice pera} a \textbf{rychlost psaní čar}.}
  \item{\textbf{Sklon pera} pomocí gyroskopu, který je také součástí MPU-6050.}
  \item{\textbf{Tlak} pomocí čtveřice tlakových senzorů Interlink Electronics FSR® 400.}
\end{itemize}
Kombinací těchto nasnímaných parametrů by bylo možné dostatečně věrohodně napodobit původní vlastnoruční podpis. 

\begin{figure}[h]
\centering
\begin{minipage}{0.45\textwidth}
    \centering
    \includegraphics[width=\textwidth]{obrazky-figures/placeholder.pdf}
    \caption{Vzhled dynamického podpisu.}
    \label{fig:first-image}
\end{minipage}\hfill
\begin{minipage}{0.45\textwidth}
    \centering
    \includegraphics[width=\textwidth]{obrazky-figures/placeholder.pdf}
    \caption{Graf dynamických parametrů podpisu v čase.}
    \label{fig:second-image}
\end{minipage}
\end{figure}


\section{Rozpoznávání falzifikátů}
Rozpoznávání falzifikátů je u digitálního podpisu automatizované.
Probíhá na základě porovnávání uloženého vzorku v databázi s podpisem, kterým se daný člověk pokouší autentizovat. %použito správně slovo autentizce?
Referenční podpis je zprůměrován z několika vzorků, aby došlo k potlačení náhodných jevů. % zav_prace 17183 2.3
Tyto vzorky jsou poskytnuté danou osobou při registraci do systému.

Porovnávací algoritmus extrahuje důležité parametry podpisu, které poté porovnává s referenčním vzorem. 
Výsledkem takového porovnávání je určité procento shody parametrů.
Následně je vypočteno celkové procento shody podpisů a na jeho základě je autentizace úspěšná či nikoli.
Je velmi důležité, aby algoritmus měl určenou správnou procentuální hodnotu shody.
Pokud by byla špatně nastavena, bylo by příliš mnoho neúspěšných autentizací, které by měly být úspěšné, nebo naopak mnoho úspěšných autentizací, které měly být neúspěšné.


\begin{figure}[h]
\centering
\includegraphics[width=0.5\textwidth]{obrazky-figures/placeholder.pdf}
\caption{Průběh autentizace}
\label{fig:my-pdf}
\end{figure}

\chapter{Analýza současných řešení}
\section{Současná řešení v~oblasti biometrie}
\section{Příklady systémů pro digitální snímání podpisu}
\section{Hodnocení výhod a nevýhod existujících řešení}

\chapter{Návrh snímacího pera}
\section{Výběr senzorů a technologií}
\subsection{EPS32}
\subsection{MPU-6050 (akcelerometr a gyroskop)}
\subsection{Tlakový senzor Interlink Electronics FSR® 400}
\section{Schéma a popis návrhu}

\chapter{Implementace prototypu}
\section{Vývoj hardwaru}
\section{Sbírání dat v~reálném čase}
\section{Ukládání dat pro následnou replikaci podpisu}

\chapter{Replikace podpisu}
\section{Využití robotické ruky}
\section{Transformace nasnímaných dat na kód pro replikaci}
\section{Testování a výsledky replikace}

\chapter{Hodnocení výsledků}
\section{Analýza dosažených výsledků}
\section{Porovnání s~očekáváními}
\section{Diskuze o~spolehlivosti a přesnosti replikace}

\chapter{Možná vylepšení a rozšíření}
\section{Návrhy na zlepšení prototypu}
\section{Možnosti dalšího výzkumu a vývoje}

\chapter{Závěr}
\section{Shrnutí hlavních poznatků}
\section{Zhodnocení významu práce}
\section{Budoucí perspektivy}

%===============================================================================

% Pro kompilaci po částech (viz projekt.tex) nutno odkomentovat
%\end{document}
