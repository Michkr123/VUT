% Tento soubor nahraďte vlastním souborem s obsahem práce.
%=========================================================================
% Autoři: Michal Bidlo, Bohuslav Křena, Jaroslav Dytrych, Petr Veigend a Adam Herout 2019

% Pro kompilaci po částech (viz projekt.tex), nutno odkomentovat a upravit
%\documentclass[../projekt.tex]{subfiles}
%\begin{document}

\chapter{Úvod}
Podpis je jednou z~nejstarších metod používanou pro ověření totožnosti, určení autorství či udělení pravního souhlasu.
V~této práci půjde především o~ověření totožnosti, které spadá pod obor zvaný biometrie.
Biometrie se zabývá analýzou biologických a behaviorálních charakteristik používaných mimo jiné k~autentizaci. 
Zkoumáním pravosti podpisu a jiných textů se zabývá písmoznalectví neboli grafognózie.
U~podpisu lze analyzovat statické a dynamické parametry písma. 

%Čím se tedy zabývá tato práce?
Podpis je v~praxi stále nejrozsáhlejším způsobem autentizace. 
Podaří-li se důveryhodně podpis napodobit, bude možnost obejít bezpečností zabezpečení. 
To představuje velké bezpečností riziko a nutnost upravení dosavadních bezpečnostích metod pro autentizaci. 
Bavíme se zde spíše o~digitálním podpisu. Ten obsahuje daleko méně informací a je tedy méně náročné ho falzifikovat. %nebo ne? lol

%TOLE JE SPIS ABSTRAKT
%Bude třeba sestavit snímací zařízení pro získaní potřebných parametrů podpisu.
%Ze statických parametrů budeme snímat tvar i tloušťku čar a celkový vzhled podpisu. Z dynamických parametrů budeme snímat tlak, sklon a průběh tahů. Získané data je potřeba vhodným způsobem zpracovat a provést potřebné výpočty. Upravená data je poté potřeba převést do instrukcí pro mechanickou paži, která je použita k replikaci podpisu. Výsledný podpis bude porovnáván s vlastnoručním podpisem. Cílem je co nejvíce se přiblížit vlastnoručnímu podpisu.

\section{Pojmy}
\label{sec:pojmy}
\begin{itemize}
  \item{\textbf{Biologické autentizační metody} - zkoumají biologické charakteristiky člověka, s~jimiž se narodí. Tyto charakteristiky má každý člověk unikátní a jsou neměnitelné. Patří mezi ně například otisky prstů, skeny sítnice či duhovky.}  
  \item{\textbf{Behaviorální autentizační metody} - porovnávají behaviorální charakteristiky související s~chováním člověka. Může například porovnávat, jakým způsobem jedinec mluví nebo jak pohybuje očima.}
  \item{\textbf{Statické parametry písma} - jde o~parametry, které nezahrnují informace o~procesu psaní podpisu. Patří mezi ně například tvar a vzhled písma, umístění podpisu na stránce nebo také tloušťka čar.}
  \item{\textbf{Dynamické parametry písma} - tyto parametry se vytahují k~samotnému průbehu psaní podpisu. Například rychlost psaní, akceleraci, tlak nebo průběh tahů.}
\end{itemize}
%\section{Představení tématu}
%\section{Cíle práce}

%\section{Struktura práce}

\chapter{Teoretické základy}
Pro plné pochopení problematiky bylo potřeba nastudovat podkaldy z~několika odvětví. 
V~této části budou shrnuty všechny potřebné informace. 
Pro snažší orientaci jsou niže rozděleny do bloků.

\section{Definice podpisu a jeho charakteristiky}
Definice podpisu je poněkud problematická. 
Podpis jako takový není v~zákoně nijak přímo definován. 
Obecně je ale bráno za podpis vlastnoruční uvedění jména a příjmení, nebo jiného jedinečného a nezaměnitelného označení. %https://www.fulsoft.cz/33/komentar-zakona-89-2012-sb-obcansky-zakonik-561-uniqueidmRRWSbk196FNf8-jVUh4EtuvCojfP1Dmx8t50X_QR_yU_dtwjPItMA/

Charakteristiky podpisu je nepřeberné množství informací, podle kterých jde jednotlivé podpisy od sebe odlišit. 
Charakteristiky se dělí na statické a dynamické~\ref{sec:pojmy}. 
Jsou ovlivňovány jak fyzickým, tak i psychologickým stavem člověka v době podpisu.
Mezi hlavní zkoumané charakteristiky patří celkový vzhled podpisu, tloušťka čar, rychlost psaní, průbeh tahů... %TODO vymyslel jsem si lol

\section{Biometrická autentizace}
Biometrická autentizace probíhá na základě biologických a behaviorálních charakteristik člověka\ref{sec:pojmy}. 
Obě tyto skupiny znaků jsou pro každého člověka unikátní a nezaměnitelné. To je důvodem, proč je lze využít k autentizaci.
Existuje spousta 
%nevim nic z hlavy :(

\section{Parametry podpisu řešené v rámci této práce}
V rámci této práce budou snímány následující parametry:
\begin{itemize}
  \item{\textbf{Celkový vzhled podpisu}}
  \item{\textbf{Pohyb propisky} při psaní pomocí akcelerometru zabudovaném v MPU-6050 ze kterého bude vypočítán celkový průběh podpisu, především pak \textbf{pozice pera} a \textbf{rychlost psaní čar}.}
  \item{\textbf{Sklon propisky} pomocí gyroskopu, který je také součástí MPU-6050}
  \item{\textbf{Přítlak} pomocí čtveřice přítlakových senzorů Interlink Electronics FSR® 400.}
\end{itemize}
Kombinací těchto nasnímaných parametrů by bylo možné dostatečně věrohodně napodobit původní vlastnoruční podpis. 

\section{Rozpoznávání falsifikátů}
Rozpoznávání falzifikátů je u digitálního podpisu automatizované.
Probíhá na základě porovnávání uloženého vzorku v databázi s podpisem, kterým se daný člověk snaží autentizovat. %použito správně slovo autentizce?


\chapter{Analýza současných řešení}
\section{Současná řešení v~oblasti biometrie}
\section{Příklady systémů pro digitální snímání podpisu}
\section{Hodnocení výhod a nevýhod existujících řešení}

\chapter{Návrh snímacího pera}
\section{Výběr senzorů a technologií}
\subsection{EPS32}
\subsection{MPU-6050 (akcelerometr a gyroskop)}
\subsection{Tlakový senzor Interlink Electronics FSR® 400}
\section{Schéma a popis návrhu}

\chapter{Implementace prototypu}
\section{Vývoj hardwaru}
\section{Sbírání dat v~reálném čase}
\section{Ukládání dat pro následnou replikaci podpisu}

\chapter{Replikace podpisu}
\section{Využití robotické ruky nebo 3D tiskárny}
\section{Transformace nasnímaných dat na kód pro replikaci}
\section{Testování a výsledky replikace}

\chapter{Hodnocení výsledků}
\section{Analýza dosažených výsledků}
\section{Porovnání s~očekáváními}
\section{Diskuze o~spolehlivosti a přesnosti replikace}

\chapter{Možná vylepšení a rozšíření}
\section{Návrhy na zlepšení prototypu}
\section{Možnosti dalšího výzkumu a vývoje}

\chapter{Závěr}
\section{Shrnutí hlavních poznatků}
\section{Zhodnocení významu práce}
\section{Budoucí perspektivy}

%===============================================================================

% Pro kompilaci po částech (viz projekt.tex) nutno odkomentovat
%\end{document}
