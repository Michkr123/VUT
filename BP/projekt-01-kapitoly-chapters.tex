% Tento soubor nahraďte vlastním souborem s obsahem práce.
%=========================================================================
% Autoři: Michal Bidlo, Bohuslav Křena, Jaroslav Dytrych, Petr Veigend a Adam Herout 2019

% Pro kompilaci po částech (viz projekt.tex), nutno odkomentovat a upravit
%\documentclass[../projekt.tex]{subfiles}
%\begin{document}

\chapter{Úvod}
Podpis je jednou z nejstarších metod používanou pro ověření totožnosti, určení autorství či udělení pravního souhlasu.
V této práci půjde především o ověření totožnosti, které řeší obor zvaný biometrie.
Biometrie se zabývá analýzou biologických a behaviorálních charakteristik používaných mimo jiné i k autentizaci. 
Zkoumáním pravosti podpisu a jiných textů se zabývá písmoznalectví neboli grafognózie.
U podpisu lze analyzovat statické a dynamické parametry písma. 

%Čím se tedy zabývá tato práce?
Podpis je v praxi stále nejrozsáhlejším způsobem autentizace. 
Podaří-li se důveryhodně podpis napodobit, bude možnost obejít bezpečností zabezpečení. 
To představuje velké bezpečností riziko a nutnost upravení dosavadních bezpečnostích metod pro autentizaci. 
Bavíme se zde spíše o digitálním podpisu. Ten obsahuje daleko méně informací a je tedy méně náročné ho falzifikovat.

%TOLE JE SPIS ABSTRAKT
%Bude třeba sestavit snímací zařízení pro získaní potřebných parametrů podpisu.
%Ze statických parametrů budeme snímat tvar i tloušťku čar a celkový vzhled podpisu. Z dynamických parametrů budeme snímat tlak, sklon a průběh tahů. Získané data je potřeba vhodným způsobem zpracovat a provést potřebné výpočty. Upravená data je poté potřeba převést do instrukcí pro mechanickou paži, která je použita k replikaci podpisu. Výsledný podpis bude porovnáván s vlastnoručním podpisem. Cílem je co nejvíce se přiblížit vlastnoručnímu podpisu.

\chapter{Pojmy}
\begin{itemize}
  \item{\textbf{Biologické autentizační metody} - zkoumají biologické charakteristiky člověka, s jimiž se narodí. Tyto charakteristiky má každý člověk unikátní a jsou neměnitelné. Patří mezi ně například otisky prstů, skeny sítnice či duhovky.}  
  \item{\textbf{Behaviorální autentizační metody} - porovnávají behaviorální charakteristiky související s chováním člověka. Může například porovnávat, jakým způsobem jedinec mluví nebo jak pohybuje očima.}
  \item{\textbf{Statické parametry písma} - jde o parametry, které nezahrnují informace o procesu psaní podpisu. Patří mezi ně například tvar a vzhled písma, umístění podpisu na stránce nebo také tloušťka čar.}
  \item{\textbf{Dynamické parametry písma} - tyto parametry se vytahují k samotnému průbehu psaní podpisu. Například rychlost psaní, akceleraci, tlak nebo průběh tahů.}
\end{itemize}
\section{Představení tématu}
\section{Cíle práce}
\section{Struktura práce}

\chapter{Teoretické základy}
\section{Definice podpisu a jeho charakteristiky}
\section{Biometrická autentizace}
\section{Dynamické parametry podpisu}
\section{Rozpoznávání falsifikátů}

\chapter{Analýza současných řešení}
\section{Současná řešení v oblasti biometrie}
\section{Příklady systémů pro digitální snímání podpisu}
\section{Hodnocení výhod a nevýhod existujících řešení}

\chapter{Návrh snímacího pera}
\section{Výběr senzorů a technologií}
\subsection{EPS32}
\subsection{MPU-6050 (akcelerometr a gyroskop)}
\subsection{Tlakový senzor Interlink Electronics FSR® 400}
\section{Schéma a popis návrhu}

\chapter{Implementace prototypu}
\section{Vývoj hardwaru}
\section{Sbírání dat v reálném čase}
\section{Ukládání dat pro následnou replikaci podpisu}

\chapter{Replikace podpisu}
\section{Využití robotické ruky nebo 3D tiskárny}
\section{Transformace nasnímaných dat na kód pro replikaci}
\section{Testování a výsledky replikace}

\chapter{Hodnocení výsledků}
\section{Analýza dosažených výsledků}
\section{Porovnání s očekáváními}
\section{Diskuze o spolehlivosti a přesnosti replikace}

\chapter{Možná vylepšení a rozšíření}
\section{Návrhy na zlepšení prototypu}
\section{Možnosti dalšího výzkumu a vývoje}

\chapter{Závěr}
\section{Shrnutí hlavních poznatků}
\section{Zhodnocení významu práce}
\section{Budoucí perspektivy}

%===============================================================================

% Pro kompilaci po částech (viz projekt.tex) nutno odkomentovat
%\end{document}
