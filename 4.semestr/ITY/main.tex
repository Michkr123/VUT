\documentclass[a4paper, twocolumn, 11pt]{article}
\usepackage{graphicx} % Required for inserting images
\usepackage[left=1.4cm, top=2.3cm, text={18.3cm, 25.2cm}]{geometry}
\usepackage{lmodern}
\usepackage[utf8]{inputenc}
\usepackage[T1]{fontenc}
\usepackage[czech]{babel}
\usepackage{amsmath}

\newtheorem{theorem}{Definice}

\title{ITY_project_2}
\author{xmicha94}
\date{March 2024}

\begin{document}
\thispagestyle{empty}
\onecolumn
\begin{center}
\Huge 
\textsc{{\Huge Vysoké učení technické v Brně \\[0.5em]}}
\textsc{{\huge Fakulta informačních technologií}}
\vfill

{\LARGE Typografie a publikování –– 2. projekt \\[0.6em]
Sazba dokumentů a matematických výrazů}
\vfill
\end{center}

{\LARGE 20024 \hfill Michálek Kryštof (xmicha94)}

\newpage
\twocolumn
\section*{Úvod}
V této úloze si vysázíme titulní stranu a kousek matematického textu, v němž se vyskytují například Definice 1 nebo rovnice (2) na straně 1. Pro vytvoření těchto odkazů používáme kombinace příkazů \verb|\label|,
\verb|\ref|, \verb|\eqref| a \verb|\pageref|. Před odkazy patří nezlomitelná mezera. Pro zvýrazňování textu se používají
příkazy \verb|\verb| a \verb|\emph|.

Titulní strana je vysázena prostředím \texttt{titlepage}
a nadpis je v optickém středu s využitím \textit{přesného} zlatého řezu, který byl probrán na přednášce. Dále jsou
na titulní straně čtyři různé velikosti písma a mezi
dvojicemi řádků textu je použito řádkování se zadanou relativní velikostí 0,5 em a 0,6 em\footnote{Použijte správný typ mezery mezi číslem a jednotkou.}.

\section{Matematický text}
Matematické symboly a výrazy v plynulém textu jsou
v prostředí \texttt{math}. Definice a věty sázíme v prostředí
definovaném příkazem \verb|\newtheorem| z balíku \texttt{amsthm}.
Tato prostředí obracejí význam \verb|\emph|: uvnitř textu
sázeného kurzívou se zvýrazňuje písmem v základním řezu. Někdy je vhodné použít konstrukci \verb|${}$|
nebo \verb|\mbox{}|, která zabrání zalomení (matematického) textu. Pozor také na tvar i sklon řeckých písmen:
srovnejte \verb|\epsilon| a \verb|\varepsilon|, \verb|\Xi| a \verb|\varXi|.
\begin{theorem}
\emph{Konečný přepisovací stroj} neboli \emph{Mea\-ly\-ho automat} je definován jako uspořádaná pětice
tvaru $M = (Q, \mathit\Sigma, \mathit\Gamma, \delta, q_0)$, kde:
\end{theorem}

\begin{itemize}
\item[$\bullet$] \textit{Q} \textit{je konečná množina} stavů,
\item[$\bullet$] $\mathit\Sigma$ \textit{je konečná} vstupní abeceda,
\item[$\bullet$] $\mathit\Gamma$ \textit{je konečná} výstupní abeceda,
\item[$\bullet$] $\delta$ : \textit{Q} $\times$ $\mathit\Sigma$ → \textit{Q} $\times$ $\mathit\Gamma$ \textit{je totální} přechodová funkce,
\item[$\bullet$] $q_0$ $\in$ \textit{Q} \textit{je} počáteční stav.
\end{itemize}

\subsection{Podsekce s definicí}

Pomocí přechodové funkce $\delta$ zavedeme novou funkci $\delta^*$
pro překlad vstupních slov u $\in$ $\mathit\Sigma^*$ do výstupních slov
$w$ $\in$ $\mathit\Gamma^*$.
\begin{theorem}
    Nechť $M = (\mathit{Q}, \mathit\Sigma, \mathit\Gamma, \delta, q_0)$ je Mea\-ly\-ho
    automat. \emph{Překládací funkce} $\delta^*$ : Q $\times$ $\mathit\Sigma^*$ $\times$  $\mathit\Gamma^*$ $\rightarrow$ $\mathit\Gamma^*$
    je pro každý stav q $\in$ Q, symbol x $\in$ $\mathit\Sigma$, slova u $\in$ $\mathit\Sigma^*$,
    $w$ $\in$ $\mathit\Gamma^*$ definována rekurentním předpisem:
\end{theorem} 

\begin{itemize}
\item[$\bullet$] $\delta^*(q,\varepsilon, w) = w$
\item[$\bullet$] $\delta^*(q,xu,w) = \delta^*(q',u,wy)$, kde $(q',y)$ = $\delta(q,x)$
\end{itemize}

\subsection{Rovnice}
Složitější matematické formule sázíme mimo plynulý
text pomocí prostředí \texttt{displaymath}. Lze umístit i více
výrazů na jeden řádek, ale pak je třeba tyto vhodně
oddělit, například pomocí \verb|\quad|, při dostatku místa
i \verb|\qquad|.
\begin{displaymath}
    g^{a_n} \notin A^{B_n} \qquad y^1_0 - \sqrt[5]{x + \sqrt[7]{y}}\qquad x > y^2 \geq y^3
\end{displaymath}

Velikost závorek a svislých čar je potřeba přizpůsobit jejich obsahu. Velikost lze stanovit explicitně,
anebo pomocí \verb|\left| a \verb|\right|. Kombinační čísla sázejte makrem \verb|\binom|.

\begin{displaymath}
    \Big|\bigcup P\Big| = \sum\limits _{\emptyset \neq X \subseteq P} (-1)^{|X|-1}\Big|\bigcap X\Big|
\end{displaymath}

\begin{displaymath}
    F_{n+1} = \binom{n}{0} + \binom{n - 1}{1} + \binom{n - 2}{2} + \cdots + \binom{[\frac{n}{2}]}{[\frac{n}{2}]}
\end{displaymath}

V rovnici (1) jsou tři typy závorek s různou \textit{explicitně} definovanou velikostí. Obě rovnice mají svisle zarovnaná rovnítka. Použijte k tomu vhodné prostředí.


\begin{eqnarray}
\bigg( \Big\{b \otimes \big[ c_1 \oplus c_2 \big] \circ a \Big\}^{\frac{2}{3}} \bigg) & = & log_z x \\
\int_a^b f(x)dx & = & -\int_b^a f(y)dy
\end{eqnarray}
V této větě vidíme, jak se vysází proměnná určující
limitu v běžném textu: $\lim_{m \to \infty} f(m)$.
Podobně je to i s dalšími symboly jako $\bigcup_{N\in M}$ N či $\sum\nolimits_{i=1}^m x_i^2$. 
S vynucením méně úsporné sazby příkazem \verb|\limits| budou
vzorce vysázeny v podobě $\lim\limits_{m \to \infty} f(m)$ a
$\sum\limits_{i=1}^m x_i^2$.

\section{Matice}
Pro sázení matic se používá prostředí \texttt{array} a závorky
s výškou nastavenou pomocí \verb|\left|, \verb|\right|.
\[
D = \left| 
\begin{array}{cccc}
    a_{11} & a_{12} & \cdots & a_{1n} \\
    a_{21} & a_{22} & \cdots & a_{2n} \\
    \vdots & \vdots & \ddots & \vdots \\
    a_{m1} & a_{m2} & \cdots & a_{mn} \\
\end{array} 
\right| 
=
\left|
\begin{array}{cc}
    x & y \\
    t & w
\end{array}
\right|
=
xw - yt
\]
Prostředí \texttt{array} lze úspěšně využít i jinde, například
na pravé straně následující rovnosti

$$ \binom{n}{k} = \left\{
\begin{array}{c l}
\frac{n!}{k(n-k)!} & \text{pro } 0 \leq k \leq n \\
0 & \text{jinak}
\end{array} \right. $$

\end{document}
